% \documentclass[10pt]{article}
\documentclass[10pt]{article}

\usepackage[french]{babel}
\usepackage[utf8]{inputenc}
\usepackage{graphicx}
\usepackage{geometry}
\usepackage{url}
% to be able to resume enumerations
\usepackage{mdwlist}
\usepackage{url}
% so as to print code
\usepackage{listings}
% various types of underlining
\usepackage{ulem}
\normalem
% math environments
\usepackage{amsmath, amsthm, amssymb}
\usepackage{setspace}
%\doublespace

\theoremstyle{plain}
\newtheorem{thm}{Théorème}[section]
\newtheorem{lem}[thm]{Lemme}

\theoremstyle{definition}
\newtheorem{defn}{Définition}[section]
\newtheorem{exmp}{Exemple}[section]

\theoremstyle{remark}
\newtheorem{rem}{Remarque}[section]


\geometry{hmargin=2cm, vmargin=2cm, lmargin=2cm, rmargin=2cm}
%\pagestyle{empty}

% Configuration
% correction
\newif\ifcorrection
%\correctiontrue
\correctionfalse


\title{Travaux dirigés : UML --- Cas d'utilisation\footnote{Si vous trouvez des erreurs,
que vous trouvez certaines phrases mal rédigées, qu'il manque ou qu'il y a trop
d'informations à certains endroits, alors, vous pouvez apporter votre
contribution via le dépôt \texttt{github} :
\texttt{https://github.com/sfourestier/enseignement}.}
\ifcorrection \\ Correction \fi}
\date{}

\begin{document}
% algorithm style
\lstset{
keywords={select,from,where,and,order,by,is,not,null,group,having,desc,union,
intersect,minus,or,as,distinct,insert,into,delete,values,commit,update,set},
keywordstyle=\bfseries,         % bold keyword
basicstyle=\small,              % print whole listing small
commentstyle=\itshape,          % default
stringstyle=\ttfamily,          % typewriter font for strings
tabsize=2,                      % pellegrini's favorite number ^^
numbers=left,                   % line numbers
numberstyle=\tiny,              % line numbers size
stepnumber=2,                   % line numbers step
numbersep=5pt,                  % line numbers sep ;)
frame=single,                   % enable frame
mathescape=false                % math on listings
}

\maketitle

\section{Cours}
% À partir du cours sur l'équitation

\section{Mise en pratique}
% À partir de l'exemple de la caisse enregistreuse du cours UML 2


% TODO supprimer cet exemple de code
% \section{Schéma entité-association}

% Vous disposez du schéma relationnel suivant :

% \begin{itemize}
%   \item Les schémas des relations :
%     \begin{verbatim}
%     ACTEUR (NUMERO_ACTEUR, NOM_ACTEUR, PRENOM_ACTEUR, NATION_ACTEUR, DATE_DE_NAISSANCE) 
%     ROLE (NUMERO_ACTEUR, NUMERO_FILM, NOM_DU_ROLE)
%     FILM (NUMERO_FILM, TITRE_FILM, DATE_DE_SORTIE, DUREE, GENRE, NUMERO_REALISATEUR)
%     REALISATEUR (NUMERO_REALISATEUR, NOM_REALISATEUR, PRENOM_REALISATEUR, NATION_REALISATEUR)
%     \end{verbatim}
%   \item Les contraintes de clés (primaires) :
%     \begin{itemize}
%       \item L'attribut \texttt{ACTEUR.NUMERO\_ACTEUR} est la clé de la relation
%         \texttt{ACTEUR}.
%       \item Le couple d'attributs (\texttt{ROLE.NUMERO\_ACTEUR},
%         \texttt{ROLE.NUMERO\_FILM}) est la clé de la relation \texttt{ROLE}.
%       \item L'attribut \texttt{FILM.NUMERO\_FILM} est la clé de la relation
%         \texttt{FILM}.
%       \item L'attribut \texttt{REALISATEUR.NUMERO\_REALISATEUR} est la clé de la
%         relation \texttt{REALISATEUR}.
%     \end{itemize}
%   \item Les contraintes d'intégrité référentielles :
%     \begin{itemize}
%       \item \texttt{ROLE.NUMERO\_FILM} référence \texttt{FILM.NUMERO\_FILM}.
%       \item \texttt{ROLE.NUMERO\_ACTEUR} référence \texttt{ACTEUR.NUMERO\_ACTEUR}.
%       \item \texttt{FILM.NUMERO\_REALISATEUR} référence
%         \texttt{REALISATEUR.NUMERO\_REALISATEUR}.
%     \end{itemize}
% \end{itemize}

% \textbf{Question 1}~\emph{Donnez le schéma entité-association correspondant à ce
% schéma relationnel\footnote{Le diagramme entité-association proposé en
% correction a été réalisé grâce au logiciel libre AnalyseSI:
% \url{https://launchpad.net/analysesi}.}.}

% \ifcorrection
% \begin{center}
% \end{picture}
% \end{center} \fi

\end{document}
