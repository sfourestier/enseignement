\documentclass[a4,12pt]{article}
\usepackage[francais]{babel}
\usepackage[T1]{fontenc}
\usepackage[margin=2cm]{geometry}
\usepackage{url}
%\voffset=-3.cm
%\hoffset=-0.4cm
%\textwidth=16.5cm
%\textheight=24.5cm
%\evensidemargin=+15mm
%\pagestyle{empty}

\title{\underline{Projet UTC504: Rencontres de basket-ball}}
\author{}
\date{}

\begin{document}
\maketitle

L'objectif du projet est de mettre en {\oe}uvre, sur un cas pratique, les
notions vues dans le module UTC504 à propos de Merise. Il s'agit de réaliser un
système d'information permettant de gérer des rencontrer de basket-ball.

Dans le cadre du module UTC504, nous nous contenterons de la partie modélisation
de ce système d'information.

Si des informations sont manquantes ou ambiguës, il est possible de fixer des
hypothèses à condition de les détailler dans le rapport. Ce projet est à
réaliser par binôme.

\subsection*{Énoncé}
Ce projet consiste à réaliser une base de données gérant les rencontres de
basket-ball entre les clubs d'une fédération. On dispose des informations
suivantes:
\begin{itemize}
  \item Un club de basket possède un bureau (les responsables) formé d'un
    président, d'un vice-président, d'un trésorier, d'un secrétaire, etc.
  \item Au sein d'un même club, il existe plusieurs catégories (ex: junior,
    cadet, benjamin, etc.). Pour une catégorie, il peut exister plusieurs
    équipes dans le même club (ex: cadet I, cadet II).
  \item Chaque équipe est composée de plusieurs joueurs. Une équipe a au moins
    un entraîneur.
  \item Un joueur a numéro de licence, un nom, un prénom, une date de naissance,
    une adresse. On veut aussi enregistrer la date de son entrée dans le club,
    les cumuls de points marqués et de fautes (depuis le début de la saison). Un
    joueur peut changer d'équipe.
  \item Pour un entraîneur, on enregistre son nom, son prénom et sa date
    d'entrée au club. Il peut entraîner plusieurs équipes du club.
  \item Une rencontre est caractérisée par une date et un numéro et se déroule
    entre deux équipes. À la fin de la rencontre, on enregistre les deux scores
    des équipes.
\end{itemize}
\ \\On veut stocker toutes ces informations dans la base, et pouvoir disposer des
informations suivantes:
\subsubsection*{Consultation}
Informations sur les clubs, les équipes, les joueurs. Les scores des matchs
joués à une date donnée, la liste des joueurs à une date donnée, la feuille du
match à une date donnée, etc.\\
Pour un club, on veut aussi le nombre de matchs gagnés, perdus ou nuls\ldots

\subsubsection*{Mise à jour}
Ajout, suppression, modification d'un joueur, d'un club, d'un match.

\subsection*{Travail demandé}
\begin{enumerate}
    \item Réalisation du modèle conceptuel pour \emph{1er octobre}.
    \item Réalisation du modèle relationnel pour le \emph{8 octobre}.
\end{enumerate}

\subsection*{Note}
Une extension de ce projet ou un devoir maison (basé sur un sujet plus adapté)
vous sera demandé pour le \emph{15 octobre} : il s'agira, à partir d'une
description de besoins, de réaliser des diagrammes de cas d'utilisation.

\end{document}
