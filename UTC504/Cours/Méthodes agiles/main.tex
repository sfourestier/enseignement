\documentclass[14pt]{beamer}

\usepackage[french]{babel}
\usepackage[T1]{fontenc}
\usepackage[utf8]{inputenc}

\newcommand\rae{$\rightarrow$ }
\newenvironment{framentitle}[1]
{
\begin{frame}
  \frametitle{#1}
}
{
\end{frame}
}


\usetheme{Singapore}
\setbeamertemplate{navigation symbols}{}
\setbeamertemplate{mini frames}{}

\title{UTC504 -- Systèmes d'Information et Bases de Données}
\subtitle{UML}
\author{Sébastien Fourestier}
\date{2022}


\begin{document}

\frame{\titlepage}

\begin{framentitle}{Bibliographie}
    \begin{itemize}
        \item Benoît Charroux, Aomar Osmani, Yann Thierry-Mieg: UML 2, Pratique
            de la modélisation
        \item Livre sur UML 2.5 du CNAM  % TODO À préciser
    \end{itemize}
\end{framentitle}

\begin{framentitle}{Correctifs}
    \begin{itemize}
        \item Ce cours est disponible sous licence libre sur ce dépôt github:\\
            \small{\url{https://github.com/sfourestier/enseignement}}
        \item[\ra] Voici pouvez :
            \begin{itemize}
                \item L'améliorer en proposant des \emph{Pull requests}
                \item Partager autour de points pouvant être améliorés en créant des
                    tickets (\emph{Issues})
            \end{itemize}
    \end{itemize}
\end{framentitle}

\begin{frame}
    \frametitle{Plan}
    \tableofcontents
\end{frame}

\AtBeginSection[]
{
  \begin{frame}<beamer>
    \frametitle{Plan}
    \tableofcontents[currentsection]
%    \tableofcontents[currentsection,currentsubsection]
  \end{frame}
}

\section{Introduction}

% présentation à partir des photos de UML 2

% prise de recul avec les retours d'expérience du livre UML 2.5 du CNAM
