\documentclass[14pt]{beamer}

\usepackage[french]{babel}
\usepackage[T1]{fontenc}
\usepackage[utf8]{inputenc}

\newcommand\rae{$\rightarrow$ }
\newenvironment{framentitle}[1]
{
\begin{frame}
  \frametitle{#1}
}
{
\end{frame}
}


\usetheme{Singapore}
\setbeamertemplate{navigation symbols}{}
\setbeamertemplate{mini frames}{}

\title{UTC504 -- Systèmes d'Information et Bases de Données}
\subtitle{UML}
\author{Sébastien Fourestier}
\date{2022}


\begin{document}

\frame{\titlepage}

\begin{framentitle}{Bibliographie}
    \begin{itemize}
        \item Benoît Charroux, Aomar Osmani, Yann Thierry-Mieg: UML 2, Pratique
            de la modélisation  % photos agiles
        \item Laurent Debrauwer, Fien Van Der Heyde: UML 2.5, Ilitiation,
            exemples et exercices corrigés
    \end{itemize}
\end{framentitle}

\begin{framentitle}{Correctifs}
    \begin{itemize}
        \item Ce cours est disponible sous licence libre sur ce dépôt github:\\
            \small{\url{https://github.com/sfourestier/enseignement}}
        \item[\ra] Voici pouvez :
            \begin{itemize}
                \item L'améliorer en proposant des \emph{Pull requests}
                \item Partager autour de points pouvant être améliorés en créant des
                    tickets (\emph{Issues})
            \end{itemize}
    \end{itemize}
\end{framentitle}

\begin{frame}
    \frametitle{Plan}
    \tableofcontents
\end{frame}

\AtBeginSection[]
{
  \begin{frame}<beamer>
    \frametitle{Plan}
    \tableofcontents[currentsection]
%    \tableofcontents[currentsection,currentsubsection]
  \end{frame}
}

\section{Introduction}

\begin{framentitle}{Historique}
    UML (Unified Modeling Language) est basé sur l'approche objet
    \begin{itemize}
        \item 1960: Simula, premier langage objet
        \item 1996: Booch, Rumbaugh, Jacobson : UML 0.9
        \item 1997: Booch, Rumbaugh, Jacobson : UML 1.0
        \item 1997: OMG (Open Management Group) : UML 1.1
        \item 2003: UML 1.5
            \begin{itemize}
                \item Possiblité de décrire des actions
            \end{itemize}
        \item 2005: UML 2.0
            \begin{itemize}
                \item Première évolution majeure depuis 1.0
                \item Nouveaux diagrammes et diagrammes existants enrichis
            \end{itemize}
        \item 2015: UML 2.5
            \begin{itemize}
                \item Possibilité de présenter les attributs et méthodes
                    héritées d'une classe
            \end{itemize}
    \end{itemize}
\end{framentitle}


\begin{framentitle}{UML}
    \begin{itemize}
        \item UML est utilisé lors de la phase d'analyse et de conception
            préliminaire des systèmes
        \item Il sert à spécifier les fonctionnalités uttendues du système
            (diagramme de cas d'utilisation et de séquence)
        \item à décrire l'architecture (diagramme de classes)
        \item La partie comportementale (diagramme d'activité et d'états) est
            moins utilisée. Souvent, une description textuelle informelle est
            utilisée, elle peut être accompagnée de ces diagrammes.
        \item Les utilisateurs utilisent surtout UML pour les phases
            préliminaires.
        \item L'objectif de l'OMG est de proposer un paradigme guidé par des
            modèles décrivant à la fois le codage, la gestion de la qualité, les
            tests et vérifications, et la production de la documentation
    \end{itemize}
\end{framentitle}


\begin{framentitle}{Le processus unifié}
    \begin{itemize}
        \item Processus de réalisation ou d'évolution de logiciels entièrement
            basé sur UML
        \item Mêmes auteurs qu'UML, mais on peut utiliser UML sans le pnocessus
            unifié
        \item Le processus unifié est piloté par les cas d'utilisation, qui
            permettent de décrire les exigences du projet
        \item Cycle de développement
            \begin{enumerate}
                \item Inspection (évaluation du projet): analyse financière,
                    principaux cas d'utilisation, ébauche d'architecture
                \item Élaboration: définition final des exigences et de
                    l'architecture du projet
                \item Construction: développement du logiciel
                \item Transition: déploiement du logiciel chez le client,
                    formation des utilisateurs
            \end{enumerate}
    \end{itemize}
\end{framentitle}

\begin{framentitle}{Le processus unifié}
    \begin{itemize}
        \item Chaque phase est détaillée par un ensemble d'activités.
        \item Une activité est un ensemble d'actions décrit par un diagramme
            d'activité
        \item Un dictionnaire très riche de modèles d'activités et de cas
            d'utilisation est fourni
        \item Principales activités:
            \begin{itemize}
                \item Modélisation des processus métiers
                \item Gestion des exigences
                \item Analyse et conception
                \item Implantation et test
                \item Déploiement
            \end{itemize}

        \item Le processus est pensé comme un processus itératif dans le sens où
            il découpe le problème global en sous-parties
        \item Autre possibliité: UML est bien adapté aux processus itératifs
            guidés par les besoins des utilisateurs, il peut donc être utilisé
            avec les méthodes agiles
    \end{itemize}
\end{framentitle}



\begin{framentitle}{Modèles}
    \begin{itemize}
        \item Un modèle est une représentation simple de la réalité
        \item Il s'exprime sous une forme simple et pratique pour le travail
        \item Lorsqu'il devient compliqué, il est souhaitable de le décomposer
            en plusieurs modèles simples
        \item Quelques usages:
            \begin{itemize}
                \item Décomposer un système complexe en sous-parties
                \item Décrire avec précision les besoins sans connaître les
                    détails du système
                \item Réduit les coûts lors du choix de l'approche permettant de
                    résoudre un problème
            \end{itemize}
    \end{itemize}
\end{framentitle}


\begin{framentitle}{Liste des diagrammes}
    \begin{itemize}
        \item Cas d'utilisation
        \item Diagramme de classes
        \item Diagramme d'interaction
        \item Diagramme d'états-transitions
        \item Diagramme d'activités
        \item TODO à compléter
    \end{itemize}
\end{framentitle}


\begin{framentitle}{Cas d'utilisation}
    \begin{itemize}
        \item Ils permettent de décrire des exigences attendues lors de la
            rédaction du cahier des charges d'un système, ou les fonctionnalités
            d'un système existant
        \item L'ensemble des cas d'utilisation d'un système contient les
            exigences fonctionnelles attendues ou existantes, les acteurs (les
            acteurs décrivent le rôle que prennent les utilisateurs du système)
            ainsi que les associations qui unissent acteurs et fonctionnalités.
            Cet ensemble détermine également les frontières du système, à savoir
            les fonctionnalités remplies par le système et celles qui lui sont
            externes
        \item Les cas d'utilisation servent de support pour les étapes de
            modélisation, de développement et de validation. Ils constituent un
            référentiel du dialogue entre les informaticiens et leurs clients et,
            par conséquent, une base pour l'élaboration au niveau fonctionnel du
            cahier des charges
    \end{itemize}
\end{framentitle}


\begin{framentitle}{Définition}
    \begin{itemize}
        \item «~Entre un acteur et le système, un cas d'utilisation décrit les
            actions et interactions liées à un objectif fonctionnel de
            l'acteur~»
        \item Nous allons l'approfondir en TD
    \end{itemize}
\end{framentitle}


\begin{framentitle}{Diagramme de classe}
    \begin{itemize}
        \item TODO compléter avec le cours de laurent-audibert sur
            developpez.com
    \end{itemize}
\end{framentitle}

\begin{framentitle}{Diagramme d'interaction}
    \begin{itemize}
        \item TODO
    \end{itemize}
\end{framentitle}

\begin{framentitle}{Diagramme d'états-transitions}
    \begin{itemize}
        \item TODO
    \end{itemize}
\end{framentitle}

\begin{framentitle}{Diagramme d'activités}
    \begin{itemize}
        \item TODO
    \end{itemize}
\end{framentitle}


